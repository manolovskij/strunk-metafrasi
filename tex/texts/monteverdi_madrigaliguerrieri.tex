\translationsetup
    {Claudio Monteverdi}
    {1638}
    {Madrigali guerrieri ed amorosi}
    {Μανώλης Κυζάλας}
    {}

Έχω αναλογιστεί πάνω στο ζήτημα των πρωτευόντων παθών και αισθημάτων του νου, καταλήγοντας πως είναι τα εξής τρία: ο θυμός, η μετριοπάθεια, και η ταπεινότητα, αυτά που κηρύττουν και οι καλύτεροι φιλόσοφοι, αλλά και η ίδια η φύση της φωνής μας το προδίδει αυτό, έχοντας τρία ξεχωριστά επίπεδα, το χαμηλό, το μεσαίο και το υψηλό. Η τέχνη της μουσικής το προδίδει επίσης, με τους όρους \emph{concitato}, \emph{molle} και \emph{temperato} (ενθουσιώδες, μαλακό και μετριοπαθές). Σε όλα τα έργα προγενέστερων συνθετών, μπόρεσα να βρω παραδείγματα molle και temperato, αλλά σε κανένα concitato, ένα είδος που περιγράφεται από τον Πλάτωνα, στο τρίτο βιβλίο του περί ρητορικής με αυτά τα λόγια: \enquote{ας πάρουμε την αρμονία που θα ταίριαζε στην μίμηση της έκφρασης και των τονισμών ενός γενναίου άνδρα που πολεμάει}. Γνωρίζοντας ήδη πως οι αντιθέσεις είναι αυτές που συγκινούν τον νου, και πως ο σκοπός της καλής μουσικής, όπως τον έθεσε ο Βοήθιος, είναι να \enquote{σχετίζεται με εμάς, εξευγενίζοντας ή διαφθείροντας τον χαρακτήρα μας}, για αυτόν τον λόγο, επιμελώς και με πολύ προσοχή, προσπάθησα να ανακαλύψω ξανά αυτό το γένος της μουσικής.

Αφού αφουγκράστηκα τους καλύτερους φιλοσόφους, που αναφέρουν πως το πυρρικό μέτρο χρησιμοποιείται για στους ενθουσιώδεις και πολεμικούς χορούς, και το αργό σπονδικό για το αντίθετο, αναλογίστηκα την αξία ενός τετάρτου, και εφάρμοσα τον έναν παλμό του σπονδικού ρυθμού να αντιστοιχεί με ένα τέταρτο· όταν αυτό διαιρέθηκε σε 16 δέκατα έκτα (μέσα σε ένα μέτρο), το ένα μετά το άλλο και συνδυάζοντας το με λόγο που εκφράζει θυμό και περιφρόνηση, κατάλαβα πως αυτό είναι το αίσθημα που έψαχνα, ακόμα και αν ο λόγος δεν ακολουθούσε την ταχύτητα των οργάνων.

Για να αποκτήσω περαιτέρω αποδείξεις, αναλογίστηκα το έργου του θείου ποιητή \en{Tasso}, που εκφράζει με μεγάλη κοσμιότητα και φυσικότητα τις ποιότητες που περιγράφει, και επέλεξα την περιγραφή του για την μάχη του \en{Tancredi} και της \en{Clorinda}, ως μία ευκαιρία να περιγράψω στην μουσική έντονα πάθη, όπως ο πόλεμος και ο θάνατος. Το 1624, αυτή η σύνθεση επιτελέστηκε στον σπουδαίο οίκο του χορηγού και καλοσυνάτου προστάτη μου, του Λαμπερότατου και Έξοχου \en{Signor Girolamo Moceningo}, διαπρεπή αξιωματούχο που υπηρετεί την Πιο Γαλήνια Δημοκρατία, με την παρουσία των εξοχώτερων και ευγενέστερων πολιτών της Βενετίας, που την δέχτηκαν με επευφημίες και επαίνους.

Μετά την επιτυχία της πρώτης μου προσπάθειας να αναπαραστήσω θυμό, συνέχισα με μεγαλύτερο πείσμα, την εξερεύνησή μου πάνω σε αυτό, και συνέθεσα και άλλα έργα αυτού του είδους, θρησκευτικά αλλά και δωματίου. Επιπλέον, αυτό το γένος ήταν τόσο θελκτικό και σε άλλους συνθέτες, που όχι μόνο δέχτηκα επαίνους με τα λόγια τους, αλλά με μεγάλη ευχαρίστηση είδα να το μιμούνται στα δικά τους έργα. Για αυτόν τόν λόγο, θεώρησα σημαντικό να αναφέρω εδώ πως η έρευνα και η πρώτη διατριβή στο γένος αυτό, τόσο σημαντικό για την τέχνη της μουσικής, έγινε από εμένα. Αξίζει να ειπωθεί, πως για αυτόν τον λόγο, η μουσική μέχρι την εποχή μας, ήταν ατελής, έχοντας μόνο τα δύο γενικά γένη· το molle και το temperato.

Αρχίκα φάνηκε παράλογο στους μουσικούς, ειδικά αυτούς που καλούνταν να παίξουν basso continuo, και όχι αξιέπαινο, το να παίξουν μία χορδή δεκαέξι φορές μέσα σε ένα μέτρο, και για αυτόν τον λόγο την έπαιζαν μόνο μία φορά, ηχώντας στο μέτρο της σπονδής αντί του πυρρικού ρυθμού, καταστρέφοντας έτσι την αίσθηση της ενθουσιώδους ομιλίας. Ας σημειωθεί, πως αυτού του είδους το basso continuo και τα συνοδευόμενα μέρη, πρέπει να παιχτούν στη μορφή και με τον τρόπο που είναι γραμμένα. Ομοίως, και στις συνθέσεις των άλλων ειδών, όλες οι γραμμένες οδηγίες είναι αναγκαίες για την επιτέλεσή τους, γιατί ο τρόπος της επιτέλεσης πρέπει να βασίζεται σε τρία πράγματα: κείμενο, αρμονία και ρυθμό.

Η επανανακάλυψή μου, αυτού του πολεμικού γένους, μου έδωσε την δυνατότητα να γράψω ορισμένα μαδριγάλια, που ονομάζω \emph{Guerriri} (πολεμικά). Η μουσική που παίζεται μπροστά σε πρίγκιπες, στις αυλές τους, έχει ως σκοπό να ικανοποιήσει το γούστο τους, που είναι τριών ειδών, σύμφωνα με τη μέθοδο της επιτέλεσης -- μουσική για θέατρο, μουσική δωματίου και χορευτική -- και για αυτό έχω σημειώσει τους τίτλους με τις λέξεις, \emph{Guerriera}, \emph{Amorosa}, \emph{Rappresentativa}.

Γνωρίζω πως αυτό το έργο μου είναι ατελές, γιατί δεν έχω αποκτήση ακόμα μεγάλη ικανότητα σε αυτό το πολεμικό γένος, αφού είναι και καινούργιο και \emph{omne principium est debile}. Για αυτό εύχομαι, ο αγαπητός αναγνώστης να δεχθεί την καλή μου θέληση, που αναμένει, από τη δική του σοφότερη πένα, έργα κοντύτερα στην τελειότητα αυτού του ύφους, αφού \emph{inventis facile est addere}. Αντίο.