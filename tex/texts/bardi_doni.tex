\translationsetup
    {Pietro de' Bardi}
    {1634}
    {Γράμμα στον G. B. Doni}
    {Μανώλης Κυζάλας}
    {Το παρακάτω γράμμα είναι από τον Pietro de' Bardi, τον γιο του Giovanni, χορηγό και προωθητή της πρώτης Φλωρεντινής Καμεράτας προς τον G. B. Doni (1594-1647), ο οποίος είχε ζητήσει να μάθει για τις πρώτες προσπάθειες πραγμάτωσης του νέου αναπαραστατικού ύφους, που ο Pietro είχε δει από κοντά στο σπίτι του πατέρα του όσο ήταν μικρός.}

    Στον πολύ επιφανή και σεβαστό χορηγό μου, τον Αξιότιμο Κύριο Giovanni Battista Doni.

Ο Πατέρας μου, κύριε Giovanni, που τον ευχαριστούσε ιδιαίτερα η μουσική, και ήταν συνθέτης κάποιας φήμης στον καιρό του, περιτριγυριζόταν από τους πιο ονομαστούς άνδρες της πόλης, γνώστες αυτού του επαγγέλματος, τους καλούσε σπίτι του και συγκροτούσε έτσι κάποιου είδους πολύ ευχάριστης και συνεχούς ακαδημίας, από όπου έλειπε κάθε κακία και ανταγωνισμός. Πολλοί ευγενείς νέοι της Φλορεντίας ήθελαν να βρεθούν να παραβρεθούν ώστε να αποκομίσουν κέρδος για τον εαυτό τους, περνώντας τον χρόνο τους όχι μόνο εξερευνώντας την μουσική, αλλά και συζητώντας και μαθαίνοντας για ποίηση, αστρολογία και άλλες επιστήμες που καθιστούσαν πολύ εποικοδομητικό τον διάλογο αυτόν.

Ο Vincenzo Galilei, ο πατέρας του διάσημου αστρονόμου των ημερών μας, άνδρας σεβαστής φήμης τότε, απασχολούταν τόσο από αυτή την έξοχη σύναξη που, συνδυάζοντας την πρακτική μουσική, στην οποία ήταν πολύ καλός, με την θεωρία της μουσικής, έκανε την προσπάθεια, με τη βοήθεια των παρευρισκόμενων βιρτουόζων και των δικών του ερευνών, να εξάγει την ουσία των ελληνικών, λατινικών και μοντέρνων κειμένων, και μέσω αυτού έγινε πολύ καλός γνώστης της θεωρίας κάθε είδους μουσικής.

Αυτός ο τόσο έξυπνος άνθρωπος αναγνώρισε πως, πέρα από την αναβίωση της αρχαίας μουσικής, όσο αυτό ήταν εφικτό, ένας από τους βασικότερους σκοπούς της ακαδημίας ήταν να βελτιώσει την μοντέρνα μουσική και να της ανυψώσει το επίπεδο, που ήταν τόσο αισχρά υποβαθμισμένο, κυρίως από τους Γότθους, μετά την εξαφάνιση της αρχαίας μουσικής και των άλλων καλων τεχνών και επιστημών. Έτσι, ήταν ο πρώτος που μας προσέφερε την ευκαιρία να ακούσουμε τραγούδι σε stile rappresentativo, και σε αυτό το πολύ δύσκολο εγχείρημα, που τότε το θεώρησαν αστείο, τον ενθάρρυνε και τον βοήθησε ο πατέρας μου, που πέρασε ολόκληρες νύχτες ξύπνιος και αφιέρωσε πολλά σε αυτήν την σπουδαία ανακάλυψη, όπως ο ίδιος ο Vincenzo αναγνωρίζει με ευγνωμοσύνη στο σπουδαίο βιβλίο του για την παλιά και την μοντέρνα μουσική. Με τον ίδιο τρόπο μας προσέφερε την ευκαιρία να ακούσουμε τους θρήνους του Count Ugolino, από τον Δάντη, από έναν καλό τενόρο, που το τραγούδησε με κατανοητό τρόπο, και ένα σύνολο από viol. Αυτή η καινοτομία, αν και έκανε εμφανή την ζήλια των επαγγελματιών, ήταν τόσο ευχάριστη για τους πραγματικούς φίλους της τέχνης. Συνεχίζοντας το εγχείρημα αυτό, ο Galilei έγραψε μουσική ένα μέρος από τους θρήνους και τα αντίφωνα της μεγάλης εβδομάδας, που τραγουδήθηκαν από ευχάριστη παρέα και με τον ίδιο τρόπο.

Ο Giulio Caccini, που θεωρείται ένας σπάνιος τραγουδιστής και άνδρας με πολύ καλό γούστο, αν και ήταν πολύ νέος, βρισκόταν τότε στην Camerata του πατέρα μου, και νιώθοντας πως του ταίριαζε τόσο η νέα αυτή μουσική, που ξεκίνησε, αποκλειστικά με τις οδηγίες του πατέρα μου, να τραγουδάει ariettas, sonnets, και άλλα ποιήματα που τους ταιριάζει η αφήγηση, με συνοδεία μόνο ενός οργάνου, και με έναν τρόπο που ενθουσίαζε το ακροατήριό του.

Επίσης στην Φλωρεντία εκείνον τον καιρό, βρισκόταν ο Jacopo Peri, που, ήταν ο πρώτος μαθητής του Cristofano Malvezzi, και έχει δεχθεί έξοχες τιμές ως εκτελεστής οργάνου και άλλων πληκτροφόρων οργάνων, αλλά και ως συνθέτης αντίστιξης, ενώ επίσης τον θεωρούσαν, πολύ σωστά, τον καλύτερο τραγουδιστή της πόλης. Αυτός, μαζί με τον Giulio, ανάδειξαν το stile rappresentativo, και αποφεύγοντας μία σκληρότητα, και μία παλαιότητα που είχε το ύφος του Galilei, κατάφεραν να κάνουν το ύφος πιο γλυκό, ώσπου ήταν ικανό να κινήσει τα συναισθήματα με έναν πολύ σπάνιο τρόπο.

Κάνοντας αυτό, απέκτησαν τον τίτλο των πρωτοπόρων και εφευρετών αυτού του είδους σύνθεσης και τραγουδιού. Ο Peri ήταν πιο επιστημονικός στην μεθοδολογία του, και είχε βρει έναν τρόπο να μιμείται τον ήχο της ομιλίας που είναι οικείος σε εμάς, χρησιμοποιώντας μόνο λίγους ήχους, και με πολλή προσοχή στην λεπτομέρεια κατάφερε να γίνει πολύ διάσημος. Του Giulio οι εφευρέσεις από την άλλη, ήταν πιο κομψές.

Το πρώτο ποίημα που τραγουδήθηκε στην σκηνή σε stile rappresentativo ήταν η ιστορία της \emph{Δάφνης}, από τον Signor Ottavio Rinuccini, που την μουσική για κάποιες μικρές σκηνές την είχε γράψει ο Peri, και τραγουδήθηκε σε ένα μικρό ιδιωτικό δωμάτιο. Είχα μείνει άναυδος από τον ενθουσιασμό μου. Τραγουδήθηκε με συνοδεία ενός συνόλου, και συνέχισαν με διασκευές πάνω σε άλλες κωμωδίες. Ο Caccini και ο Peri ένιωθαν μεγάλη ευγνωμοσύνη προς τον Signor Ottavio, αλλά ακόμα μεγαλύτερη προς τον Signor Jacopo Corsi, που έγινε πολύ σκληρός και κακεντρεχής με τα πάντα εκτός από τις πιο ακραίες μορφές αυτής της τέχνης, ο οποίος όμως καθοδήγησε αυτούς τους συνθέτες με εξαιρετικές ιδέες και λαμπερά διδάγματα, όπως ενδείκνυται για ένα τόσο σπουδαίο εγχείρημα. Η καθοδήγησή του πληρώνεται από τον Peri και τον Caccini σε όλες τις συνθέσεις τους, με πολλούς διαφορετικούς τρόπους.

Μετά από την \emph{Δάφνη}, πολλές αναπαραστάσεις πραγματώθηκαν από τον Signor Ottavio Rinuccini, που, σαν καλός μουσικός αλλά και ποιητής, είχαν μεγάλη απήχηση, όπως είχε και ο ερειστικός Corsi, που υποστήριξε το εγχείρημα πολύ γενναιόδωρα. Οι πιο δημοφιλείς από αυτές τις αναπαραστάσεις έγιναν για τις ιστορίες της \emph{Ευριδίκης} και της \emph{Αριάννας}· πέρα από αυτές, πολλές μικρότερες διασκευάστηκαν από τον Caccini και τον Peri. Δεν υπήρξε κάποια διάθεση να τους μιμηθούν, ούτε εδώ στην Φλωρεντία, την πατρίδα αυτής της μουσικής, ούτε σε άλλη πόλη της Ιταλίας, ιδιαίτερα στην Ρώμη, που έχουν τα δικά τους λαμπρά παραδείγματα στα δράματα. Ανάμεσά τους μου φαίνεται αναγκαίο πως θα πρέπει να αναφέρω τον Monteverdi.

Πολύ φοβάμαι πως δεν εκτέλεσα σωστά, Αξιοσέβαστε Άρχοντα μου, την εντολή σας, όχι μόνο επειδή άργησα πολύ να την φέρω εις πέρας, αλλά και επειδή δεν μπόρεσα να ικανοποιήσω τον εαυτό μου, διότι έχουν μείνει λόγοι που θυμούνται πια εκείνη τη μουσική εκείνων των καιρών. Παρ' όλα αυτά, πιστέυω να εξυπηρέτησα τις επιθυμίες σας, με την περιορισμένη επιλογή μου, από όλα τα πράγματα που μπορούν να ειπωθούν για το ύφος της \emph{musica rappresentativa} που εκτιμάτε τόσο πολύ.

Ελπίζω όμως, να με συγχωρείτε με την ευγένειά σας, Αξιότιμε Άρχοντα μου, και σας εύχομαι τα πιο ευτυχισμένα χριστούγεννα. Προσεύχομαι στον Θεό, να σας χαρίσει την μέγιστη ευτυχία.

\begin{flushleft}
    Φλωρεντία, Δεκεμβριος 14, 1634\\
    Της Σπουδαίας Αξιότιμης Αρχοντιάς σας\\
    ο πιο Ταπεινός υπηρέτης\\
    Pietro Bardi, Ceonte di Vernio
\end{flushleft}