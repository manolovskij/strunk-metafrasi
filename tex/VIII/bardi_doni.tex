\translationsetup
    {Pietro de' Bardi}
    {1634}
    {Γράμμα στον G. B. Doni}
    {Μανώλης Κυζάλας}
    {Το παρακάτω γράμμα είναι από τον Pietro de' Bardi, τον γιο του Giovanni, χορηγό και προωθητή της πρώτης Φλωρεντινής Καμεράτας προς τον G. B. Doni (1594-1647), ο οποίος είχε ζητήσει να μάθει για τις πρώτες προσπάθειες πραγμάτωσης του νέου αναπαραστατικού ύφους, που ο Pietro είχε δει από κοντά στο σπίτι του πατέρα του όσο ήταν μικρός.}

    Στον πολύ επιφανή και σεβαστό χορηγό μου, τον Αξιότιμο Κύριο Giovanni Battista Doni.

Ο Πατέρας μου, κύριε Giovanni, που το ευχαριστούσε ιδιαίτερα η μουσική, και ήταν συνθέτης κάποιας φήμης στον καιρό του, περιτριγυριζόταν από τους πιο ονομαστούς άνδρες της πόλης, γνώστες αυτού του επαγγέλματος, τους καλούσε σπίτι του και συγκροτούσε έτσι κάποιου είδους πολύ ευχάριστης και συνεχούς ακαδημίας, από όπου έλειπε κάθε κακία και ανταγωνισμός. Πολλοί ευγενείς νέοι της Φλορεντίας ήθελαν να βρεθούν να παραβρεθούν ώστε να αποκομίσουν κέρδος για τον εαυτό τους, περνώντας τον χρόνο τους όχι μόνο εξερευνώντας την μουσική, αλλά και συζητώντας και μαθαίνοντας για ποίηση, αστρολογία και άλλες επιστήμες που καθιστούσαν πολύ εποικοδομητικό τον διάλογο αυτόν.

Ο Vincenzo Galilei, ο πατέρας του διάσημου αστρονόμου των ημερών μας, άνδρας σεβαστής φήμης τότε, απασχολούταν τόσο από αυτή την έξοχη σύναξη που, συνδυάζοντας την πρακτική μουσική, στην οποία ήταν πολύ καλός, με την θεωρία της μουσικής, έκανε την προσπάθεια, με τη βοήθεια των παρευρισκόμενων βιρτουόζων και των δικών του ερευνών, να εξάγει την ουσία των ελληνικών, λατινικών και μοντέρνων κειμένων, και μέσω αυτού έγινε πολύ καλός γνώστης της θεωρίας κάθε είδους μουσικής.

% This great intellect recognized that, besides restoring ancient music in so far as so obscure a subject permitted, one of the chief aims of the academy was to improve modern music and to raise it in some degree from the wretched state to which it had been reduced, chiefly by the Goths, after the loss of the ancient music and of the other liberal arts and sciences. Thus he was the first to let us hear singing in stile rapprcscnta- tivo, in which arduous undertaking, then considered almost ridiculous, he was chiefly encouraged and assisted by my father, who toiled for en- tire nights and incurred great expense for the sake of this noble discovery, as the said Vincenzo gratefully acknowledges to my father in his learned book on ancient and modern music.2 Accordingly he let us hear the lament of Count Ugolino, from Dante,8 intelligibly sung by a good tenor and precisely accompanied by a consort of viols. This novelty, al.. though it aroused considerable envy among the professional musicians) was pleasing to the true lovers of the art. Continuing with this under- taking, Galilei set to music a part of the Lamentations and Responds of Holy Week, and these were sung in devout company in the same man- ner.

% Giulio Caccini, considered a rare singer and a man of taste, although very young, was at this time in my father's "Camerata," and feeling himself inclined toward this new music, he began, entirely under my father's instructions, to sing ariettas, sonnets, and other poems suitable for reading aloud, to a single instrument and in a manner that astonished his hearers.

% Also in Florence at this time was Jacopo Peri, who, as the first pupil of Cristofano Malvezzi, received high praise as a player of the organ and the keyboard instruments and as a composer of counterpoint and was rightly regarded as second to none of the singers in that city. This man) in competition with Giulio, brought the enterprise of the stile rapprsssnta... t;vo to light, and avoiding a certain roughness and excessive antiquity which had been felt in the compositions of Galilei, he sweetened this style, together with Giulio, and made it capable of moving the passions in a rare manner, as in the course of time was done by them both.

% By so doing, these men acquired the title of the first singers and in- ventors of this manner of composing and singing. Peri had more science, and having found a way of imitating familiar speech by using few sounds and by meticulous exactness in other respects, he won great fame. Giulio's inventions had more elegance.

% The first poem to be sung on the stage in stile ral'l'resentativo was the story of Dalne, by Signor Ottavio Rinuccini, set to music by Peri in few numbers and short scenes and recited and sung privately in a small room.4 I was left speechless with amazement. It was sung to the accom- paniment of a consort of instruments, an arrangement followed there- after in the other comedies. Caccini and Peri were under great obliga- tion to Signor Ottavio, but under still greater to Signor Jacopo Corsi," who, becoming ardent and discontent with all but the superlative in thIS art, directed these composers with excellent ideas and marvelous doc- trines, as befitted so noble an enterprise. These directions were carried out by Peri and Caccini in all their compositions of this sort and were com- bined by them in various manners.

% After the Dafnc, many stories were represented by Signor Ottavio himself, who, as good poet and good musician in one, was received with great applause, as was the affable Corsi, who supported the enterprise with a lavish hand. The most fanlous of these stories were the Euridicc and the Arianna; 8 besides these, many shorter ones were set to music by Caccini and Peri. Nor was there any want of men to imitate them, and in Florence, the first home of this sort of music, and in other cities of Italy, especially in Rome, these gave and are still giving a marvelous account of themselves on the dramatic stage. Among the foremost of these it seems fitting to place Monteverdi.

% I fear that I have badly carried out Your Most Reverend Lordship's command, not only because I have been slow to obey Your Lordship, but also because I have far from satisfied myself, for there are few now living who remember the music of those times. Nonetheless I believe that as I serve Your Lordship with heartfelt affection, so Your Lordship will confirm the truth of my small selection from the many things that might be said about this style of musica rappresentativa which is in such esteem.

% But I hope that I shaJl in some way be excused through the kindness of Your Most Excellent Lordship, and predicting for Your Lordship a most happy Christmas, I pray that God Himself, the father of all blessings, may grant Your Lordship perfect felicity.

% Florence, December 14, 1634.

% Your Very Illustrious and Reverend Lordship's

% Most humble servant,

% Pietro Bardi, Conte di Vernia.